\documentclass[a4paper,twoside,11.5pt,reqno]{article}

\usepackage{amssymb}
\usepackage{ragged2e}
\usepackage{amsmath}
\usepackage{amsthm}
\usepackage{natbib}
\usepackage{setspace}
\usepackage{authblk}
\usepackage{graphicx}
\usepackage{caption}
\usepackage{subcaption}
\usepackage{booktabs}
\usepackage{multirow}
\usepackage{float}
\usepackage{color}
\usepackage{soul}
\usepackage[flushleft]{threeparttable}
%\usepackage{subfig}

\newtheorem{theorem*}{Theorem}
\newtheorem{Prop}{Proposition}

\renewcommand{\vec}[1]{\mathbf{#1}}


%Page set up space
\setlength{\oddsidemargin}{-0.125in}
\setlength{\evensidemargin}{-0.125in}
\setlength{\topmargin}{-0.25in}
\setlength{\textwidth}{6.5in}
\setlength{\textheight}{9.25in}
\setlength{\parskip}{\medskipamount}
\setlength{\parindent}{0pt}

\setstretch{1.25}


\begin{document}

\title{Supplementary Material for Replication: Focality and Asymmetry in Multi-battle Contests}
\author[1]{Subhasish M. Chowdhury}
\author[2]{Dan Kovenock}
\author[3]{\\David Rojo Arjona}
\author[4]{Nathaniel T. Wilcox}

%Affiliation
\affil[1]{\scriptsize Department of Economics, University of Bath, Bath BA2 7AY, UK.}
\affil[2]{\scriptsize Economic Science Institute, Chapman University, One University Drive, Orange, CA 92866, USA.}
\affil[3]{\scriptsize Smith Institute, Chapman University, One University Drive, Orange, CA 92866, USA.}
\affil[4]{\scriptsize Department of Economics, Appalachian State University, Boone, NC 28608, USA.}

\date{\today}

\maketitle

\section{Description of Replication Package}
This replication package includes the following files:

\begin{itemize}
\item A copy of the IRB approval for the research project to which our experiment is associated: IRBApproval.pdf.
\item A copy of the experimental design: Experimental Design.pdf.
\item A copy of the experimental instructions: Experimental Instructions.pdf.
\item The raw experimental dataset in non-proprietary format: RawData.csv.
\item The raw experimental dataset in Stata format: RawData.dta.
\item Do file for replication of the analysis in the main paper, and Appendix: Results.do (Stata 16.1 used).
\item Mathematica files for best-response and $\epsilon$-equilibrium Figures: BestResponses.nb and Epsilon.nb (Mathematica 10.0 used)
\end{itemize}

\section{Description of the raw experimental dataset}
List and description of the variables:
\begin{itemize}
\item treatment: Unique Treatment Identifier $1 =IS$; $2 =IF$; $3 =IV$; $4 =IL$; $5 =AS$; $6 = AF$; $7 =AV$.
\item session: Unique Session Identifier.
\item id: Unique Subject Identifier in Experiment.
\item period: Period.
\item player: Unique Subject Identifier in Session.
\item partner: Opponent's player number.
\item tokens: Budget.
\item partnertokens: Opponent's budget.
\item decisionlengthms: Decision Time Length (ms).
\item allocation$X$: Allocation to battlefield $X$, where $X=1$ is the target box. 
\item partnerallocation$X$: Opponent's allocation to battlefield $X$.
\item earnings$X$: Realized earnings in battlefield $X$.
\item battle$X$value: Value of Battlefield $X$.
\item battle$X$location: Location of Battlefield $X$, where values go from $1 =$ ``Far Left'' to $4 =$ ``Far Right''.
\item battle$X$color: Color of Battlefield $X$, where $0 =$ ``White'' and $1 =$ ``Black''.
\item win$X$: Indicator variable $1 =$ ``Victory in Battlefield $X$'' and $0 =$ ``Otherwise''.
\end{itemize}

\section{Description of the do file}
The Stata do file ``Results.do'' contains the code to replicate the figures and analysis of the experimental data. Figures are saved on the working directory and Tables are displayed on the screen in a matrix named after the corresponding Table number (e.g., matrix `Table3' corresponds to Table 3 in the main text). The running time for the whole code is few seconds in Stata 16.1.

\section{Description of the Mathematica files}
The Mathematica file ``BestResponses.nb'' contains the code to replicate Figure 2 and Figure A1. The Mathematica file ``Epsilon.nb'' contains the code to replicate Figure D1.
\end{document}